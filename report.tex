\documentclass{scrreprt}


\begin{document}
\section{Letter from the GNOME Foundation}

karen

It is bittersweet to be introducing the 2013 GNOME annual report. This financial year was my last year as GNOME's Executive Director, as I left the position in March of 2014. However, I'm thrilled to have recently been elected to GNOME's Board of Directors and am happy to continue to contribute to GNOME's future. GNOME is such an important, vibrant project, and I feel lucky to be able to play a part in it.

As you will see when you read this annual report, there have been a lot of great things that have happened for the GNOME Foundation during this period. Two new companies joined our advisory board, the Linux Foundation and Private Internet Access.  The work funded by our accessibility campaign was completed and we ran a successful campaign for privacy. During this period, there was a fantastic Board of Directors, a dedicated Engagement team (who worked so hard to put this report together), and the conference teams (GNOME.Asia, GUADEC and the Montreal Summit) knocked it out of the park. Most importantly, we’ve had an influx of contributors, more so than I’ve seen in some time.
I hope that I have helped us to get in touch with our values during my time as Executive Director and I think that GNOME is more aware of its guiding mission than ever before. The ongoing success of the Outreach Program for Women (OPW) and positive relations with other organizations fighting for software freedom have all helped us to tell a powerful story about who we are and why we matter.
There's a lot of work ahead of us as we catch up organizationally to the successes of OPW and as we work to make GNOME the best it can be, but there is no better community of people to accomplish this.
I’m excited for the future and can’t wait to see where we go next.

karensig

Karen Sandler, GNOME Executive Director (June 2011-March 2014)



imgp2917


\section{Hackfests}

Much of the development of GNOME takes place over the internet. This makes face-to-face meetings an invaluable opportunity for GNOME contributors to collaborate, have fun, get to know each other, and get work done. The following is a chronological account of the hackfests which took place in 2013.


\subsection{Developer Experience - Brussels, Belgium}

The goal of this productive hackfest was to improve the GNOME 
application developer experience. Attendees split into groups in order 
to address and analyze the following areas: application distribution \& 
sandboxing, documentation, toolkit, and development tools. The 
documentation group worked on creating application development, as well 
as the design of the developer documetation website. Those in the 
development tools group addressed Devhelp's user interface, and also 
made progress on generating documentation for Javascript and GObject 
Introspection. The toolkit group pushed forward with important new 
features such as GtkFlowBox, GNotification, the popular GtkHeaderBars, 
GtkPopovers, and accessibility integration with Clutter.

Docs Hackfest - Brno, Czech Republic
At Devconf, a yearly conference organized by Red Hat in the Czech Republic, the Documentation Team focused on updating GNOME's help pages in time for the GNOME 3.8 release. In addition, new help documentation for the System Monitor and GNOME Terminal were finalized and work began on writing help pages for Boxes. Some progress was also made with developer documentation.

Freedesktop Summit - Nuremberg, Germany
The Freedesktop Summit is a joint technical meeting for those involved in shared infrastructure for the major Free Software desktops. Developers from KDE, GNOME, Unity, and the Razor-qt projects met at the SUSE offices in Nuremberg, Germany, to improve collaboration. Topics addressed during this meeting included D-Bus specifications for application launching, kdbus, and a replacement for X11-based startup notification.

GNOME Fest - Lima, Peru
Peruvian GNOME contributors organized a conference about GNOME in Lima. Over 100 students attended the event which featured a variety of activities and talks on the GNOME community, programs, coding tutorials, using GNOME with Arduino, and more.

GTK+ - Boston, USA
Apart from an interruption due to a city-wide lockdown in Boston, this hackfest was very productive. A range of elements in GTK+ were finalized and important plans for the future were discussed. Preliminary plans were made for using Wayland in GNOME Shell, and the roadmap for HiDPI support in GTK+ was established. Additionally, significant work was done implementing new components for GNOME 3 applications such as GtkListBox, GtkFlowBox, and GtkHeaderBar.

Marketing - New York, USA
The GNOME Marketing Team (now rebranded as the GNOME Engagement Team) used this hackfest to build a foundation for the team.  The beginning of the hackfest focused on  fundamental questions such as: What is GNOME? Why does GNOME matter?  The later part of the hackfest focused on the visual identity and brand presentation for GNOME.

Open Help Docs - Cincinnati, OH, USA
Following the 2013 OpenHelp conference, members of the GNOME Documentation Team convened to brainstorm ideas on improving GNOME developer documentation. Among the things the team worked on during this hackfest were a revamp the GNOME Platform Overview as well as application development documentation in general. A guide for preparing applications for translation was also produced, in addition to Vala examples in the platform tutorials. Finally, the new HowDoI developer documentation initiative was started, which encourages GNOME developers to write specific tutorial-style guides for common tasks.

.NET + GNOME - Vienna, Austria
During the .NET + GNOME hackfest, attendees worked on modernizing various C\# GNOME apps and technologies. In particular, work was done to port Banshee, Pinta, SparkleShare, and Tomboy to GTK\#3. Also, some work was done in creating versions of Tomboy for Android and OS X, as well as updating the GTK\#3 bindings to support GTK+ 3.10.

Maps - Gothenburg, Sweden
The GNOME Maps hackfest was a small event which afforded two GSoC students the opportunity to meet their mentor and collaborate with designers in person. Various map-related GNOME technologies and designs were discussed, plans were made for GNOME Maps, and some development was done on Geoclue2.

GNOME Summit - Montréal, Canada
The annual GNOME Summit typically takes place in Boston. However, in 2013 it changed location to Montréal, Canada. Despite the change of location, the outcome was very productive. The new GNotification API was completed and there was some brainstorming on design as well as functionality of GNOME Boxes. Lastly, a handful of accessibility regressions were fixed, and plans were made to ensure proper accessibility support in GNOME under Wayland.

WebKitGTK+ - A Coruña, Galicia, Spain
With 30 people in attendance, the fifth year of the WebKitGTK+ hackfest was the largest to date. A variety of areas was worked on during this event, including Wayland support for WebKit2 and WebKitGTK+, design and functional improvements to GNOME Web, porting the build system to CMake, and improving the integration of the new Web Inspector with WebKitGTK+.


gnome-yay



hackfest




\section{Conferences}

$9022723355_2190e1bce7_k$


GNOME.Asia Summit
GNOME.Asia Summit 2013 was held in Seoul, the capital of South Korea, from May 23 to May 24. Many people passionate about GNOME came to Seoul from China, Taiwan, Japan, Hong Kong, Indonesia, USA, Canada, France, the UK, Germany, India, and elsewhere around the world.
The local team and GNOME.Asia Summit committee spent more than 6 months preparing for the conference. Many thanks go to the Korean government's National IT Industry Promotional Agency (NIPA) who offered the use of their venue and assisted with organization.
Karen Sandler and Allan Day keynoted the conference. Karen talked about her pacemaker which runs proprietary software and inspired us by reminding us of the need for Free Software in our everyday lives. Allan Day talked about the history and future of GNOME 3. He highlighted the progress GNOME had made over the past year and the new features which would come soon.
There were talks on a wide range of topics including GNOME technologies such as Rygel and GStreamer, as well as input methods. A training session on translating GNOME was also run. The audience awarded the best session with a prize generously sponsored by Lemote.

Día GNOME
Día GNOME (GNOME Day) 2013 was held in Temuco, Chile on November 9. The main topics included GNOME 3.10, Ubuntu GNOME, how to write good bug reports, and how to write apps with PyGobject. Other activities also took place including “Olimpiadas GNOME” (GNOME Olympics), a fun sport event, and “Guess the Movie”, an entertaining trivia event where the winners received GNOME t-shirts and stickers. Around 50 people participated in the event. 




GUADEC
GUADEC is the largest event for GNOME users and developers, held annually in Europe. In 2013, GUADEC was held in Brno, Czech Republic, from August 1 to 8.
Hundreds of contributors participated in GUADEC 2013, including volunteers, interns, and the employees of many companies. The schedule included talks on a wide range of topics -- technological developments and plans, design, and community outreach. The GNOME community also used the conference to meet with partners and to make plans for the future, including new GTK+ features, Wayland support, new geolocation infrastructure, and application sandboxing/bundling.
We had the pleasure to have keynotes from Ethan Lee, Matt Dalio, Cathy Malmrose, and our very own GNOME Foundation Board of Directors.
Ethan Lee spoke about the challenges of porting games to GNU/Linux and how we can help by providing better tools and community support.
Matt Dalio from Endless Mobile opened the second day of the conference by talking about his plans to use GNOME technologies to bring computers to people around the world who currently lack access to them.
On day three, the GNOME community got the opportunity to join a question and answers session with the newly elected GNOME Foundation Board of Directors.
Finally, on the last core day of GUADEC, Cathy Malmrose talked about her company, ZaReason, which sells computers pre-installed with GNU/Linux.
During GUADEC 2013, we also had the chance to hear about all the amazing work that has been done by our interns, during their lightning talks. A large audience gathered to hear about the many projects that have been undertaken by interns in the Outreach Program for Women and Google Summer of Code.
The final days of GUADEC were very productive with numerous BoFs and hackfests, as is tradition. Sessions were held on documentation, marketing, translation and accessibility, as well as on development areas such as input methods, Pitivi, Evolution, Wayland porting, GTK+, and geolocation.
Overall, GUADEC 2013 included 42 talks, two sessions of lightning talks, two parties, 15 working sessions, three hackfests. Social events included a football match, a city tour, and a Creative Commons film night.









\section{Internship programs}


GNOME continued to grow its intership and outreach activities in 2013 through its participation in Google Summer of Code  as well as its organization of the Outreach Program for Women. Both programs attracted new contributors and gave them an opportunity to gain skills and experience working in Free Software. This included a variety of areas, such as programming, design, documentation, and marketing.
In total, GNOME supported 31 interns through Google Summer of Code, all of whom worked on programming projects for the summer.
The Outreach Program for Women (OPW) has been organized by the GNOME Foundation since 2010. The initiative pairs interns with mentors from a number of Free Software projects and provides them with a stipend to work on a fixed-term internship. In 2013 the program was highly successful and grew to include 62 interns from a number of organizations and companies, including the GNOME Foundation, Google, Intel, Linux Foundation, Mozilla, Fedora, JBoss, Perl, the OpenStack Foundation, the Free Software Foundation, MediaGoblin, Red Hat, Wikimedia Foundation, WordPress, the Yocto Project, OpenMRS, Subversion, and Tor. Fifteen interns worked on the GNOME project as a part of OPW in 2013.
GUADEC was a major opportunity for interns to make contact with their mentors and the rest of the GNOME project. 22 Google Summer of Code interns joined us in Brno in the summer of 2013, alongside 8 OPW participants. Special events took place for interns, which aimed to help them integrate and gain confidence, including lightning talks and a social event. The GNOME Foundation also produced a year book for all interns, which was distributed during the conference.




\section{Interns in 2013}

Aakanksha Gaur
Aakash Goenka
Alessandro Campagni
Alex Muñoz
Anton Belka
Aruna Sankaranarayanan
Bogdan Gabriel Ciobanu
Camilo Polymeris
Carlos Soriano
Dylan McCall
Eslam Mostafa
Evgeny Bobkin
Flavia Weisghizzi
Garima Joshi
Gökcen Eraslan
Guillaume Mazoyer
Joris Valette
Kalev Lember
Lavanya Gunasekaran
Magdalen Berns
Marcos Chavarría Teijeiro
Mathieu Duponchelle
Mattias Bengtsson
Meg Ford
Melissa S.R. Wen
Parin Porecha
Poeteris Krijanis
Pooja Saxena
Rafael Fonseca
Richard Schwarting
Sai Suman Prayaga
Sam Bull
Satabdi Das
Saumya Dwivedi
Saumya Pathak
Sébastien Wilmet
Shivani Poddar
Simon Corsin
Sindhu S
Tiffany Yau
Ting-Wei Lan
Tomasz Maczynski
Valentín Barros
Victor Toso
Xuan Hu
Žan Doberšek





summerofcode


$9414435399_889cb0e2bd_o$


\section{Finances}

The GNOME Foundation 2013 financial year ran from 1st October 2012 to 30th September 2013. Main sources of income for the GNOME Foundation in the 2012-13 financial year included Advisory Board fees, OPW sponsorship, corporate sponsorship and private donations. The main outgoings were employees, OPW and events such as conferences and hackfests. In 2013, outgoings exceeded income by approximately \$80 000. This shortfall was primarily a result of funds that were previously set aside for a system administrator, as well as OPW invoices that were paid later than expected.
Administrative expenses were on the rise in 2013, as the Foundation sent out an increased number of reimbursements. To reduce these fees in 2014, the Foundation now uses a different bank which charges less to send and receive international payments. Administration expenses also included costs associated with running web services and purchasing office supplies and hardware. Employee costs were higher in 2013 as the Foundation contracted Andrea Veri to work on system and service administration using funds that were raised for this purpose.
The main GUADEC expense in 2013 was travel and accommodation sponsorship: the Foundation sponsored a total of 54 attendees for GUADEC 2013. \$5 000 was received in GSoC royalties, and just over \$2 000 in Amazon referral fees and royalties on merchandise sales. The OPW was the highest expense in 2013, but most of these expenses are expected to be covered by corporate sponsorship. GNOME sponsored two interns in 2013, one in the 5th round and one in the 6th round of the program.

income table

expense table



\section{Accessibility}

$9445513292_04018d0557_k$


Accessibility is a core value of the GNOME project and its mission is to bring Free Software to everyone. In 2013, the Accessibility Team was hard at work making GNOME more accessible, thus enabling more people to enjoy GNOME.
Improvements were made to keyboard navigation and GNOME's Universal Access settings were redesigned. Also, many new additions to GTK+, such as popover widgets, were given accessibility support, and Orca received a significant performance boost.
Two long-awaited accessibility features landed in GNOME in 2013. The first was the addition of caret and focus tracking, which simplifies keyboard navigation while the magnifier is in use. The second was the addition of PDF caret and keyboard navigation in Evince, which allows Orca users to read documents in both Evince and GNOME Documents. This work was carried out by Igalia and was funded by the Friends of GNOME Accessibility Campaign and the Mozilla Foundation.
The GNOME Project continued its commitment to 'built-in' rather than 'bolted-on' accessibility in 2013. The work mentioned above, but also many more improvements which are not listed here, landed in the 3.8 and 3.10 releases and clearly showed how valuable GNOME's efforts are.
Plans were also made to advance accessibility in the future. Structural and semantic information was made available through Poppler and, according to the plans,  will be used to further improve acccessibility in Evince. Other plans included making sure that GNOME on Wayland continues to be accessible.



\section{Privacy campaign}
privacy

Due to events such as CISPA, computer users' privacy was a topic of concern in 2013 and received considerable industry and media attention. Users have been increasingly concerned about their own privacy and the extent that they are being exposed to and exploited by software companies and government agencies.
As a Free Software project dedicated to users' overall experience, the GNOME project is in a unique position to implement features that strengthen users' privacy. This was highlighted by the computer security researcher Jacob Appelbaum at GUADEC in 2012. Running a fundraiser centered around privacy was the next logical step for the GNOME Foundation.
GNOME's privacy campaign was launched in December 2012 and successfully collected \$20,000  during the following 7 months. These funds will be used to strengthen and implement new privacy features in GNOME. The conclusion of the campaign doesn't represent the end of the Foundation's privacy efforts - it is only the beginning. The GNOME Foundation welcomes any individual who would like to make a difference by working on features in GNOME which strengthen privacy.
As a non-profit charity, the GNOME Foundation is dependent on its financial supporters. We encourage parties to either join the Advisory Board or make an individual donation as part of the Friends of GNOME program.



\section{Bugzilla statistics}

figure

2011

2012

2013


table

2011   2012  2013




Bugs Closed         Patches Contributed     Bugs Reported       Patches Reviewed



\section{GNOME Releases}
apps-view

2013 included two GNOME releases: 3.8 in March, and 3.10 in September. Both releases included new features as well as general improvements to GNOME 3. Both of the new versions marked a significant improvement for GNOME's user and developer experiences.
Highlights for 3.8 included new application launching and search views, new Privacy and Sharing settings, improved animation and video rendering, and input methods integration. A large proportion of GNOME's system settings were reworked, a new initial setup assistant was introduced, and Web, the GNOME browser, was upgraded to WebKit2.
3.10 also contained many features and improvements. Initial Wayland support was introduced, as well as a new combined system status menu. Many applications were updated to use the new header bar widget, and a raft of new applications were introduced, including Maps, Software, Notes, Music, and Photos. GNOME 3.10 also included a new geolocation framework.
High-resolution display support, Software, and Wayland are three of the most exciting features from the 3.8 and 3.10 releases.
High Resolution Display Support
Displays with high resolutions have become increasingly common in the past few years. Screens like this require that interface toolkits adjust their resolution to compensate. High resolution display support has been lacking in the Free Software desktop space, and GNOME was the first project to introduce it in the 3.10 release of 2013.
Work in this area was greatly assisted by the donation of a number of high resolution laptops to the GNOME Foundation: first, by Brion Vibber, an individual supporter, and later by Intel's Open Source Technology Center.
Software
Software is GNOME's new application for installing applications and managing software updates. It aims to provide an "app store"-like experience, which makes it easy to find applications to install, either through recommendations, ratings or browsing by category.
As a part of this effort to provide a more modern application installation experience, GNOME has been working with upstream applications to ensure that they provide the necessary metadata.
Wayland
Wayland is the next generation technology for display and input on Linux. It promises to deliver smoother graphics, with improved animations and transitions. Its modern architecture will provide greater flexibility for developers and will enable more secure sandboxed applications.
3.10 introduced experimental Wayland support, which provided the ability for developers to test GNOME running on Wayland, and provided the basis for further development work. To do this, GNOME contributors have been working closely with the Wayland development team and are helping to shape the future of the Linux graphics stack.

Further details about the GNOME 3.8 and 3.10 releases can be found in our release notes.



new-apps



guadec-group

\section{Advisory Board}

The Advisory Board is made up of organizations and companies that support GNOME. Advisory Board membership helps support the overall infrastructure for GNOME and its members communicate with the Board of Directors, helping them to guide the direction of GNOME and the Foundation. The Advisory Board has no decision-making authority but provides a vehicle for its members to communicate with the Board of Directors and help the Directors guide the overall direction of GNOME and the GNOME Foundation.
The Advisory Board consists of representatives from the following GNOME Foundation member corporations and projects:

canonical

intel

collabora

sflc

google

redhat

mozilla

sugarlabs

olpc

ibm

igalia

suse

debian

pia

linuxfoundation

fsf

Without the support of these companies and organizations, many of GNOME's activities in 2013 would not have been possible.


\section{Friends of GNOME}

Thank you to everyone who donated in 2013!

Adam Byrtek
Adam Dingle
Adrian Boldi
Adrian Spirgi
Alan Morgan
Albert Gasset Romo
Albert Hopkins
Albert Vernon
Alberto Salmerón Moreno
Alessandro Mecca
Alex Converse
Alex Muñoz
Alexandre H Abdo
Alexandre Savio
alibek junisbayev
Alishams Hassam
Amjad Al Taleb
Anderson Goulart
Andre Luis Gobbi Sanches
Andreas Altergott
Andreas Nilsson
Andreas Rugtved Neumann
Andrew Burrow
Andrew Rabon
Andrey Ivanov
Andriy Kusov
Arnaud Mounier
Arno Teigseth
Baptiste Mille-Mathias
bastiaan van der veer
Bastian Hougaard
Behdad Esfahbod
Benjamin Lebsanft
Bertel King
Bertrand Lorentz
Billy Harris
Blaise Alleyne
Botond Denes
Bowie Poag
Brian Fagioli
Brian Visel
Bruce Reimel
Bruno de Mello
Bryen Yunashko
Carles Guadall Blancafort
Carlos A Iglesias
Carlos Antonio Marquês Maniero
Carlos sepulveda mancill
Carsten Olsen
Cecile Veneziani
Cedric Martinez Campos
Che-Hsun Liu
Christian Lucas
Christian Meißner
Christine Spang
Christoph Ulbrich
Christopher Hanson
Christopher Ludwig
Christopher Meiklejohn
Clemens Zeitlhofer
Craig Keogh
Cristián Rojas
Daniel Aleksandersen
Daniel Doel
Daniel Hogan
Daniel Landau
Daniel Pinske
Daniel Rodriguez
Daniel Thompson
David Norman
Denis Andrade
Denis Donici
Diego Toral
Dillon Gilmore
Dillon Gilmore
Dmitry Kabanov
Dmitry Kleva
Dominic Janczak
Dor Tzur
Dr. Michael Darmer
Edgar Jimenez
Eduard Drenth
Edward Jakus
Einar J Haraldseid
Elizabeth Gossett
Emily Gonyer
Emmanuele Bassi
Ernesto Gutierrez
Evan Leister
Fabio Castelli
Fanen Ahua
Florian Sowade Florian Sowade
Francisco de la Peña
Frank Zequim
Gavin Ferris
Gilles Crieloue
Grégoire Seux
Gregory Wellington
Guilherme Mesquita Gondim
Hajime Mizuno
Hans Hellsten
Hassan Sunbul
Helio Albano de Oliveira
Ilja Sekler
Ilya Litvinov
Iván dominguez martin
Jacob Larsen
Jaime Velázquez Sánchez
James Campbell
James Cook
James M Jinkins
James Mason
Jan Girlich
Jan Leike
Jan Szpuk
Jan-Christoph Borchardt
Jarl Frode Arntzen
Jason Jarquín Sevilla
Jason Kelsey
Jason Weill
Jean-Peer Lorenz
Jenny Morgan
Jérôme Perret
Jesús Espino García
Jimmy Richards
Joan Cervan i Andreu
Joaquim Gil Hoernecke
Joel Burleson
Joel Luellwitz
Johannes Schmid
John Conkell
Jonathan Barnoud
Jonathan MOREL
JOONE HUR
jorge castro
Jorge Gallegos
José Emanuel Dávila Alanís
JOSE LUIS LOPEZ DE CIORDIA
Jose Maria Casanova Crespo
Jose Miguel Dana Perez
Joseph Braddock
Joshua Melling
js darnell
Juan Garcia
Juan Jose Marin Martinez
Julie Pichon
Julien Thuillier
Karol Babioch
Kerry Chhim
Kristian Tizzard
Krzysztof Krzyzaniak
Ladislav Morva
Laurent Goujon
Leif Gruenwoldt
Leo Hnatek
Leslie Chen
Linus Seelinger
Luca Daghino
Luis Villa
Luiz Fernando Silva
Magne Larsen
Mahendra Tallur
Manish Sinha
Marat Dyatko
Marc Schröder
Marc-Andre Lureau
Marco Bollero
Marina Zhurakhinskaya
Marius Gedminas
Mark Lee
Mark Pariente
Markus Griesslehner
Martin Ansdell-Smith
Martin Budsjö
Marvin Munguia
Mathias Nicolajsen Kjaergaard
Matj Tý
Matteo Settenvini
Matthew Lee
Matthieu Coudert
Michael Blennerhassett
Michael Jakobsen
Michael Lissner
Michael Loney
Michel De Waele
Michel Machado
Miguel Lorenzo Amarelle
Mike Williamson
Mikel Olasagasti Uranga
Mirsal Ennaime
Morgan Dapilly
Nathan Erickson
Nick Jennings
Nik Henry
Nikolai Neff
Oktay Acikalin
Pablo Estigarribia Davyt
Pascal Terjan
Patrick Verner
Patrizio Bruno
Paul Reust
Pedro de Medeiros
Peter Bui
Peter Ulber
Petr Volkov
Petter Johansson
Philippe Gauthier
Prabowo Saputro
Rasmus Pedersen
Richard Bodo
Rob Middleton
Robert Allgeyer
Robert McCallum
Robin Stocker
Roger Lancefield
Roxana Murgan León
Ruben Solvang
RUI Gouveia
Ruslan Zhenetl
Russell Sim
Salomon Sickert
Saulo Machado de Souza Jacques
Scott Mcdonald
Sean Anastasi
Sean Brady
Sebastien Grenier
Serdar Cizmeci
Shushi Kurose
Shwan Ciyako
Simon Engelbert
Simon Wenner
Stanislaw Kulczycki
Stefan Lehmann
Stefan Schindler
Stéphane Démurget
stijn Van Campenhout
Suresh Kadthan
Susan Roelofs
Terence Honles
Thomas Bechtold
Thomas Bollmeier
Thomas Eberhardt
Thomas Heidrich
Thomas Jenkins
Thomas Lowenthal
Tiffany Antopolski
Tomas Östlund
Torsten Kirschner
Ukasz Jerna
Vadzim Rutkouski
Vigneshwaran P
Vinicius Seixas
Vítzslav Vojtchovský
Willem van der Schyff
William Lazenby
William McClary
Wolf Vollprecht
Young Lee
Zigurds Gavars
Zwahlen Joël



\section{Credits}

\subsection{Authors}

Allan Day
Andreas Nilsson
David King
Ekaterina Gerasimova
Emily Gonyer
Fabiana Simões
Fabio Duran 
Hashem Nasarat
Juanjo Marin
Karen Sandler
Oliver Propst
Peri Helion
Rosanna Yuen
Tobias Mueller



\subsection{Design}
Allan Day
Andreas Nilsson

\subsection{Photos}

this should be automatically done

page 2
Ekaterina Gerasimova CC BY-NC-SA 2.0

page 4
All rights reserved Allan Day
All rights reserved ChangSeok Oh

page 5
All rights reserved Fabio Durán Verdug

page 6
Sammy Fung CC BY-NC-SA 2.0

page 9
Jakub Steiner CC BY-SA 2.0

page 7, 11 16
Ana Rey CC BY-SA 2.0

\end{document}
