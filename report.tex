\documentclass{scrreprt}


\begin{document}
\section{Letter from the GNOME Foundation}

karen

It is bittersweet to be introducing the 2013 GNOME annual report. This financial year was my last year as GNOME's Executive Director, as I left the position in March of 2014. However, I'm thrilled to have recently been elected to GNOME's Board of Directors and am happy to continue to contribute to GNOME's future. GNOME is such an important, vibrant project, and I feel lucky to be able to play a part in it.

As you will see when you read this annual report, there have been a lot of great things that have happened for the GNOME Foundation during this period. Two new companies joined our advisory board, the Linux Foundation and Private Internet Access.  The work funded by our accessibility campaign was completed and we ran a successful campaign for privacy. During this period, there was a fantastic Board of Directors, a dedicated Engagement team (who worked so hard to put this report together), and the conference teams (GNOME.Asia, GUADEC and the Montreal Summit) knocked it out of the park. Most importantly, we’ve had an influx of contributors, more so than I’ve seen in some time.
I hope that I have helped us to get in touch with our values during my time as Executive Director and I think that GNOME is more aware of its guiding mission than ever before. The ongoing success of the Outreach Program for Women (OPW) and positive relations with other organizations fighting for software freedom have all helped us to tell a powerful story about who we are and why we matter.
There's a lot of work ahead of us as we catch up organizationally to the successes of OPW and as we work to make GNOME the best it can be, but there is no better community of people to accomplish this.
I’m excited for the future and can’t wait to see where we go next.

karensig

Karen Sandler, GNOME Executive Director (June 2011-March 2014)



imgp2917


\section{Hackfests}

Much of the development of GNOME takes place over the internet. This makes face-to-face meetings an invaluable opportunity for GNOME contributors to collaborate, have fun, get to know each other, and get work done. The following is a chronological account of the hackfests which took place in 2013.


\subsection{Developer Experience - Brussels, Belgium}

The goal of this productive hackfest was to improve the GNOME 
application developer experience. Attendees split into groups in order 
to address and analyze the following areas: application distribution \& 
sandboxing, documentation, toolkit, and development tools. The 
documentation group worked on creating application development, as well 
as the design of the developer documetation website. Those in the 
development tools group addressed Devhelp's user interface, and also 
made progress on generating documentation for Javascript and GObject 
Introspection. The toolkit group pushed forward with important new 
features such as GtkFlowBox, GNotification, the popular GtkHeaderBars, 
GtkPopovers, and accessibility integration with Clutter.

Docs Hackfest - Brno, Czech Republic
At Devconf, a yearly conference organized by Red Hat in the Czech Republic, the Documentation Team focused on updating GNOME's help pages in time for the GNOME 3.8 release. In addition, new help documentation for the System Monitor and GNOME Terminal were finalized and work began on writing help pages for Boxes. Some progress was also made with developer documentation.

Freedesktop Summit - Nuremberg, Germany
The Freedesktop Summit is a joint technical meeting for those involved in shared infrastructure for the major Free Software desktops. Developers from KDE, GNOME, Unity, and the Razor-qt projects met at the SUSE offices in Nuremberg, Germany, to improve collaboration. Topics addressed during this meeting included D-Bus specifications for application launching, kdbus, and a replacement for X11-based startup notification.

GNOME Fest - Lima, Peru
Peruvian GNOME contributors organized a conference about GNOME in Lima. Over 100 students attended the event which featured a variety of activities and talks on the GNOME community, programs, coding tutorials, using GNOME with Arduino, and more.

GTK+ - Boston, USA
Apart from an interruption due to a city-wide lockdown in Boston, this hackfest was very productive. A range of elements in GTK+ were finalized and important plans for the future were discussed. Preliminary plans were made for using Wayland in GNOME Shell, and the roadmap for HiDPI support in GTK+ was established. Additionally, significant work was done implementing new components for GNOME 3 applications such as GtkListBox, GtkFlowBox, and GtkHeaderBar.

Marketing - New York, USA
The GNOME Marketing Team (now rebranded as the GNOME Engagement Team) used this hackfest to build a foundation for the team.  The beginning of the hackfest focused on  fundamental questions such as: What is GNOME? Why does GNOME matter?  The later part of the hackfest focused on the visual identity and brand presentation for GNOME.

Open Help Docs - Cincinnati, OH, USA
Following the 2013 OpenHelp conference, members of the GNOME Documentation Team convened to brainstorm ideas on improving GNOME developer documentation. Among the things the team worked on during this hackfest were a revamp the GNOME Platform Overview as well as application development documentation in general. A guide for preparing applications for translation was also produced, in addition to Vala examples in the platform tutorials. Finally, the new HowDoI developer documentation initiative was started, which encourages GNOME developers to write specific tutorial-style guides for common tasks.

.NET + GNOME - Vienna, Austria
During the .NET + GNOME hackfest, attendees worked on modernizing various C\# GNOME apps and technologies. In particular, work was done to port Banshee, Pinta, SparkleShare, and Tomboy to GTK\#3. Also, some work was done in creating versions of Tomboy for Android and OS X, as well as updating the GTK\#3 bindings to support GTK+ 3.10.

Maps - Gothenburg, Sweden
The GNOME Maps hackfest was a small event which afforded two GSoC students the opportunity to meet their mentor and collaborate with designers in person. Various map-related GNOME technologies and designs were discussed, plans were made for GNOME Maps, and some development was done on Geoclue2.

GNOME Summit - Montréal, Canada
The annual GNOME Summit typically takes place in Boston. However, in 2013 it changed location to Montréal, Canada. Despite the change of location, the outcome was very productive. The new GNotification API was completed and there was some brainstorming on design as well as functionality of GNOME Boxes. Lastly, a handful of accessibility regressions were fixed, and plans were made to ensure proper accessibility support in GNOME under Wayland.

WebKitGTK+ - A Coruña, Galicia, Spain
With 30 people in attendance, the fifth year of the WebKitGTK+ hackfest was the largest to date. A variety of areas was worked on during this event, including Wayland support for WebKit2 and WebKitGTK+, design and functional improvements to GNOME Web, porting the build system to CMake, and improving the integration of the new Web Inspector with WebKitGTK+.



\end{document}
